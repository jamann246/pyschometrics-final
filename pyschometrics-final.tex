% Options for packages loaded elsewhere
\PassOptionsToPackage{unicode}{hyperref}
\PassOptionsToPackage{hyphens}{url}
\PassOptionsToPackage{dvipsnames,svgnames,x11names}{xcolor}
%
\documentclass[
  letterpaper,
  DIV=11,
  numbers=noendperiod]{scrartcl}

\usepackage{amsmath,amssymb}
\usepackage{setspace}
\usepackage{iftex}
\ifPDFTeX
  \usepackage[T1]{fontenc}
  \usepackage[utf8]{inputenc}
  \usepackage{textcomp} % provide euro and other symbols
\else % if luatex or xetex
  \usepackage{unicode-math}
  \defaultfontfeatures{Scale=MatchLowercase}
  \defaultfontfeatures[\rmfamily]{Ligatures=TeX,Scale=1}
\fi
\usepackage[]{times}
\ifPDFTeX\else  
    % xetex/luatex font selection
\fi
% Use upquote if available, for straight quotes in verbatim environments
\IfFileExists{upquote.sty}{\usepackage{upquote}}{}
\IfFileExists{microtype.sty}{% use microtype if available
  \usepackage[]{microtype}
  \UseMicrotypeSet[protrusion]{basicmath} % disable protrusion for tt fonts
}{}
\makeatletter
\@ifundefined{KOMAClassName}{% if non-KOMA class
  \IfFileExists{parskip.sty}{%
    \usepackage{parskip}
  }{% else
    \setlength{\parindent}{0pt}
    \setlength{\parskip}{6pt plus 2pt minus 1pt}}
}{% if KOMA class
  \KOMAoptions{parskip=half}}
\makeatother
\usepackage{xcolor}
\usepackage[top=1in,bottom=1in,left=1in,right=1in]{geometry}
\setlength{\emergencystretch}{3em} % prevent overfull lines
\setcounter{secnumdepth}{-\maxdimen} % remove section numbering
% Make \paragraph and \subparagraph free-standing
\ifx\paragraph\undefined\else
  \let\oldparagraph\paragraph
  \renewcommand{\paragraph}[1]{\oldparagraph{#1}\mbox{}}
\fi
\ifx\subparagraph\undefined\else
  \let\oldsubparagraph\subparagraph
  \renewcommand{\subparagraph}[1]{\oldsubparagraph{#1}\mbox{}}
\fi


\providecommand{\tightlist}{%
  \setlength{\itemsep}{0pt}\setlength{\parskip}{0pt}}\usepackage{longtable,booktabs,array}
\usepackage{calc} % for calculating minipage widths
% Correct order of tables after \paragraph or \subparagraph
\usepackage{etoolbox}
\makeatletter
\patchcmd\longtable{\par}{\if@noskipsec\mbox{}\fi\par}{}{}
\makeatother
% Allow footnotes in longtable head/foot
\IfFileExists{footnotehyper.sty}{\usepackage{footnotehyper}}{\usepackage{footnote}}
\makesavenoteenv{longtable}
\usepackage{graphicx}
\makeatletter
\def\maxwidth{\ifdim\Gin@nat@width>\linewidth\linewidth\else\Gin@nat@width\fi}
\def\maxheight{\ifdim\Gin@nat@height>\textheight\textheight\else\Gin@nat@height\fi}
\makeatother
% Scale images if necessary, so that they will not overflow the page
% margins by default, and it is still possible to overwrite the defaults
% using explicit options in \includegraphics[width, height, ...]{}
\setkeys{Gin}{width=\maxwidth,height=\maxheight,keepaspectratio}
% Set default figure placement to htbp
\makeatletter
\def\fps@figure{htbp}
\makeatother
% definitions for citeproc citations
\NewDocumentCommand\citeproctext{}{}
\NewDocumentCommand\citeproc{mm}{%
  \begingroup\def\citeproctext{#2}\cite{#1}\endgroup}
\makeatletter
 % allow citations to break across lines
 \let\@cite@ofmt\@firstofone
 % avoid brackets around text for \cite:
 \def\@biblabel#1{}
 \def\@cite#1#2{{#1\if@tempswa , #2\fi}}
\makeatother
\newlength{\cslhangindent}
\setlength{\cslhangindent}{1.5em}
\newlength{\csllabelwidth}
\setlength{\csllabelwidth}{3em}
\newenvironment{CSLReferences}[2] % #1 hanging-indent, #2 entry-spacing
 {\begin{list}{}{%
  \setlength{\itemindent}{0pt}
  \setlength{\leftmargin}{0pt}
  \setlength{\parsep}{0pt}
  % turn on hanging indent if param 1 is 1
  \ifodd #1
   \setlength{\leftmargin}{\cslhangindent}
   \setlength{\itemindent}{-1\cslhangindent}
  \fi
  % set entry spacing
  \setlength{\itemsep}{#2\baselineskip}}}
 {\end{list}}
\usepackage{calc}
\newcommand{\CSLBlock}[1]{\hfill\break\parbox[t]{\linewidth}{\strut\ignorespaces#1\strut}}
\newcommand{\CSLLeftMargin}[1]{\parbox[t]{\csllabelwidth}{\strut#1\strut}}
\newcommand{\CSLRightInline}[1]{\parbox[t]{\linewidth - \csllabelwidth}{\strut#1\strut}}
\newcommand{\CSLIndent}[1]{\hspace{\cslhangindent}#1}

\KOMAoption{captions}{tableheading}
\makeatletter
\@ifpackageloaded{caption}{}{\usepackage{caption}}
\AtBeginDocument{%
\ifdefined\contentsname
  \renewcommand*\contentsname{Table of contents}
\else
  \newcommand\contentsname{Table of contents}
\fi
\ifdefined\listfigurename
  \renewcommand*\listfigurename{List of Figures}
\else
  \newcommand\listfigurename{List of Figures}
\fi
\ifdefined\listtablename
  \renewcommand*\listtablename{List of Tables}
\else
  \newcommand\listtablename{List of Tables}
\fi
\ifdefined\figurename
  \renewcommand*\figurename{Figure}
\else
  \newcommand\figurename{Figure}
\fi
\ifdefined\tablename
  \renewcommand*\tablename{Table}
\else
  \newcommand\tablename{Table}
\fi
}
\@ifpackageloaded{float}{}{\usepackage{float}}
\floatstyle{ruled}
\@ifundefined{c@chapter}{\newfloat{codelisting}{h}{lop}}{\newfloat{codelisting}{h}{lop}[chapter]}
\floatname{codelisting}{Listing}
\newcommand*\listoflistings{\listof{codelisting}{List of Listings}}
\makeatother
\makeatletter
\makeatother
\makeatletter
\@ifpackageloaded{caption}{}{\usepackage{caption}}
\@ifpackageloaded{subcaption}{}{\usepackage{subcaption}}
\makeatother
\ifLuaTeX
  \usepackage{selnolig}  % disable illegal ligatures
\fi
\usepackage{bookmark}

\IfFileExists{xurl.sty}{\usepackage{xurl}}{} % add URL line breaks if available
\urlstyle{same} % disable monospaced font for URLs
\hypersetup{
  pdftitle={Psychometric Analysis of the National Household Survey on Disaster Preparedness Affinity},
  pdfauthor={Jordan Amann},
  colorlinks=true,
  linkcolor={blue},
  filecolor={Maroon},
  citecolor={Blue},
  urlcolor={Blue},
  pdfcreator={LaTeX via pandoc}}

\title{Psychometric Analysis of the National Household Survey on
Disaster Preparedness Affinity}
\author{Jordan Amann}
\date{}

\begin{document}
\maketitle
\begin{abstract}
Natural disasters pose significant risks to physical and mental health
as well as financial stability. Household-level preparedness can be
life-saving, yet there is limited psychometric analysis of tools
measuring disaster preparedness affinity. This study analyzes the
psychometric properties of the 2023 National Household Survey (NHS),
conducted by FEMA, which includes items designed to measure preparedness
affinity. Key findings indicate good reliability and criterion validity,
but limited construct validity. Exploratory factor analysis supports a
single-factor or two-factor model representing preparedness affinity.
Further refinement of the scale is recommended to improve its
psychometric robustness.
\end{abstract}

\setstretch{2}
\subsection{Introduction}\label{introduction}

The United Nations Office for Disaster Risk Reduction defines disasters
as a serious disruption of the functioning of a community or a society
at any scale due to hazardous events interacting with conditions of
exposure, vulnerability and capacity, leading to one or more of the
following: human, material, economic and environmental losses and
impacts (United Nations Office for Disaster Risk Reduction (UNDRR)
2024). Noteably, as global climate change progresses, it will increase
the probability of extreme and hazardous weather events (Keim 2008). A
critical component of mitigating the human, material, and economic cost
of disasters is increasing government, community, and individual
preparedness. As such, understanding the degree of individual
preparedness and its driving factors is essential.

The Federal Emergency Management Agency (FEMA) has conducted an annual
National Household Survey (NHS) to measure household disaster
preparedness actions and attitudes since 2013. In the public release of
the 2023 results from the survey four concepts, made up of 7 items
included in the survey, were identified as contributing to disaster
preparedness affinity: awareness of information, disaster experience,
preparedness efficacy, and disaster risk perception (Federal Emergency
Management Agency (FEMA) 2023). These items have not been formally
defined as a scale to measure individual preparedness affinity, but it
is worth exploring the psychometric qualities of them as a scale due to
the scale and resources behind the larger survey they are delivered in.
This study aims to assess the internal consistency and reliability of
these NHS items related to disaster preparedness affinity, evaluate the
criterion validity of these items based on self-reported preparedness
actions, assess construct validity using economic and household
demographic data, and explore the latent structure of the preparedness
affinity construct using exploratory factor analysis.

\subsection{Methods}\label{methods}

\paragraph{Sampling}\label{sampling}

The 2023 NHS was a cross-sectional survey that targeted adults aged 18
and older who were U.S. residents, with a total of 7,604 complete
responses. It was distributed online via invite, and was administered
confidentially. Populations at higher risk for specific hazards (7\%,
n=525) and underrepresented groups, including American Indian, Alaska
Native, Native Hawaiian, and Pacific Islanders (7\%, n=517) were
over-sampled. Additional demographic characteristics such as age,
gender, race, ethnicity, and income were monitored while the survey was
live and compared to soft quotas which were created using U.S. Census
Bureau data from the 2021 Five-Year American Community Survey and the
2020 Decennial Census of Island Are. Seven percent (n=525) of
participants completed the survey in Spanish, all other participants
completed the survey in English. All responses included in this analysis
were complete; however, some demographic variables were imputed by FEMA,
including education, gender, race, disability, home-ownership, income
and ethnicity. Across all of these variables, the average change from
imputation was 0.51\%.

\paragraph{Measurement}\label{measurement}

Preparedness affinity was measured using a subset of items from the NHS,
which were identified by FEMA as being relevant driving factors of this
construct. The specific items can be found in Table~\ref{tbl-items}.

\paragraph{Reliability and Validity}\label{reliability-and-validity}

Psychometric analysis was conducted to evaluate reliability using
Cronbach's alpha for internal consistency, criterion validity by
correlating the preparedness affinity scale with self-reported
preparedness actions, and construct validity through correlations with
demographic variables such as income, disability status, household size,
home-ownership, age, and rural status.

\paragraph{Exploratory Factor
Analysis}\label{exploratory-factor-analysis}

Exploratory Factor Analysis using Principal Axis Factoring was used to
assess the latent structure, with scree plots and eigenvalues guiding
the selection of one-, two-, and four-factor models.

\subsection{Results}\label{results}

The preparedness affinity scale achieved a Cronbach's alpha of 0.75,
indicating good internal consistency. Preparedness affinity demonstrated
a moderately strong correlation (r = 0.67) with the total number of
preparedness actions taken, supporting good criterion validity. However,
no substantial evidence of convergent construct validity was found, as
correlations with demographic variables were negligible (r \textless{}
0.2), suggesting limited alignment with external constructs. Exploratory
Factor Analysis indicated that a single-factor model explained 36\% of
the variance with strong factor loadings across all items. A two-factor
model explained 43\% of the variance with reduced complexity and was
interpreted as ``Disaster Awareness'' and ``Risk Perception'' with
Cronbach's alphas of 0.67 and 0.68, respectively. The four-factor model
failed to produce distinct factors, with overlapping loadings and
Cronbach's alphas as low as 0.46 for individual factors.

\subsection{Discussion}\label{discussion}

The NHS preparedness affinity items demonstrated strong reliability and
acceptable criterion validity but lacked robust construct validity.
Factor analysis suggests that the preparedness affinity construct may be
best represented by a single factor, although a two-factor model also
provides a meaningful interpretation. The scale's reliability decreases
when disaggregated into multiple factors, indicating redundancy among
the hypothesized components. Comprehensive sampling ensured
representation of high-risk populations, and initial evidence supports
the use of NHS items as a psychometric instrument for preparedness
affinity.

\subsection{Limitations}\label{limitations}

The cross-sectional design of the NHS precluded assessment of
test-retest reliability, and construct validity analysis was limited by
the available demographic variables.

\subsection{Conclusion}\label{conclusion}

The 2023 NHS provides a promising foundation for measuring disaster
preparedness affinity. However, further refinement is needed to enhance
its validity and applicability in broader contexts. Future research
should develop and validate a more comprehensive model of preparedness
affinity, include additional measures for content and construct
validity, and conduct longitudinal studies to evaluate test-retest
reliability.

References

Appendix

\begin{longtable}[]{@{}
  >{\raggedright\arraybackslash}p{(\columnwidth - 2\tabcolsep) * \real{0.3521}}
  >{\raggedright\arraybackslash}p{(\columnwidth - 2\tabcolsep) * \real{0.6479}}@{}}
\caption{Preparedness Affinity Items}\label{tbl-items}\tabularnewline
\toprule\noalign{}
\endfirsthead
\endhead
\bottomrule\noalign{}
\endlastfoot
Label & Question \\
dis\_prep & Have you considered preparing for a disaster? \\
dis\_soc & Thinking about preparing yourself for a disaster, which of
the following best represents your degree of preparedness? \\
dis\_exp & Have you or your family ever experienced the impacts of a
disaster? \\
dis\_awareness & In the past year, what information have you read, seen,
or heard about how to get better prepared for a disaster? \\
dis\_perception & Thinking about the area you live in, how likely would
it be for a disaster to impact you? \\
dis\_stepshelp & How much would taking steps to prepare help you get
through a disaster in your area? \\
dis\_confidence & How confident are you that you can take steps to
prepare for a disaster in your area? \\
\end{longtable}

\phantomsection\label{refs}
\begin{CSLReferences}{1}{0}
\bibitem[\citeproctext]{ref-fema2023nhs}
Federal Emergency Management Agency (FEMA). 2023. {``{2023 National
Household Survey}.''}
\url{https://www.fema.gov/about/research-methodology/national-household-survey}.

\bibitem[\citeproctext]{ref-keim2008}
Keim, Mark E. 2008. {``Building Human Resilience.''} \emph{American
Journal of Preventive Medicine} 35 (5): 508--16.
\url{https://doi.org/10.1016/j.amepre.2008.08.022}.

\bibitem[\citeproctext]{ref-undrr_disaster_terminology}
United Nations Office for Disaster Risk Reduction (UNDRR). 2024.
{``Terminology: Disaster.''}
\url{https://www.undrr.org/terminology/disaster}.

\end{CSLReferences}



\end{document}
